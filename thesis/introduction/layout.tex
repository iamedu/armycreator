%layout
\section{Disposici\'on del documento}

En la búsqueda de una solución se desarrollaron una variedad de tareas a lo largo de este trabajo, que se han dividido en capítulos para facilitar su consulta.

El capítulo \ref{ch:about} presenta los motivos para llevar a cabo este trabajo, se discute la problemática actual para llevar a cabo el mismo, y se describe brevemente el enfoque que se tomó para darle solución.

El capítulo \ref{ch:tecnologia_arm} describe la tecnología ARM; cuál es su papel en el mercado actual, sus características y que puede hacerse con ella.

El capítulo \ref{ch:herramientas_arm} contiene una breve presentación de las herramientas de trabajo actuales que podrían ayudar a dar solución al problema planteado.

El capítulo \ref{ch:arquitectura_arm} se detalla la tecnología ARM. Para trabajar con esta tecnología es necesario un estudio de la misma, aquí se puede encontrar una descripción de sus características más relevantes para el trabajo como son el funcionamiento del procesador, los registros y la interfaz de memoria.

En el capítulo \ref{ch:solution} se describe la solución en la que se ha trabajado. Se describen los distintos componentes que se usaron para cumplir con nuestro objetivo y la arquitectura de los mismos.

En el capítulo \ref{ch:simulador_arm} se discute la arquitectura de uno de los principales componentes de éste trabajo, el simulador ARMYC. Se describe su arquitectura, como trabaja y las ideas principales detrás de su funcionamiento. Por último se muestra un caso práctico de cómo el simulador puede ayudar en el diseño de hardware de una manera sencilla.

En el capítulo \ref{ch:arm_linux} se muestra el trabajo realizado para portar Linux a una nueva arquitectura, en este caso un caso específico de la arquitectura ARM.

En el capítulo \ref{ch:tarjeta} se muestra el diseño hasta el ruteo de una plataforma de hardware ARM base, que se puede utilizar como un inicio para otros proyectos que deseen utilizar la arquitectura ARM.

Por último, en el capítulo \ref{ch:resultados} se describen los resultados obtenidos, se presenta la discusión, el trabajo a futuro y se presentan las conclusiones obtenidas con este trabajo.


