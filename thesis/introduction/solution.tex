%Solucion
\section{Soluci\'on}

Se ha desarrollado un simulador modular de la arquitectura ARM. La innovación de este simulador sobre los que ya se encuentran disponibles es la facilidad para describir dispositivos de hardware en desarrollo, o aún en fase de diseño. Gracias a este simulador es posible detectar fallas y problemas desde una etapa temprana del proyecto, también permite iniciar el desarrollo de software sin tener un prototipo terminado pues se puede portar de una manera sencilla al prototipo final sin mucho problema.

Otro punto importante para facilitar el desarrollo de ARM, es proveer un diseño de hardware inicial, desde el cual se pueda iniciar el desarrollo de un nuevo proyecto. Para este punto se ha llevado a cabo el diseño en eagle, el cual se puede usar como base para seguir con este desarrollo.

