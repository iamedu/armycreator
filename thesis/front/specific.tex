\cleardoublepage
\begingroup

\begin{addmargin}[-1cm]{-3cm}
\begin{center}

\large

\hfill

\includegraphics[height=2cm]{img/ipn.jpg}
\hfill
\includegraphics[width=2cm]{img/escom.png}

\begingroup
\color{Maroon}
INSTITUTO POLITÉCNICO NACIONAL
\\
ESCUELA SUPERIOR DE CÓMPUTO
\endgroup

\small

\emph{No. registro:} TT20090039 \hfill Serie: Amarilla \hfill Mayo de 2010

\vspace{0.5 cm}

\normalsize

\textbf{Sistema Integrado diseñado para el manejo de video en 2 dimensiones}

\vspace{1.0 cm}

Presentan

\small

\textbf{
Díaz Real Eduardo \footnote{iamedu@gmail.com} \\
Hernández Contreras Edgar \footnote{edgar.raiden@gmail.com} \\
Loyola Diaz Karina Jessica \footnote{shesicka@gmail.com} \\
Serna Romero Mauricio \footnote{boikot11@gmail.com} \\
}

\vspace{0.5 cm}

Directores

\textbf{
Dr. Miguel Ángel Alemán Arce\\
Dr. Amilcar Meneses Viveros
}

\vspace{1.0 cm}

Resumen

\end{center}


El uso de dispositivos personales se ha vuelto necesario y con ello han surgido varias tecnologías que prometen cumplir con ésta necesidad, la desventaja de muchas de estas tecnologías es que no permiten hacer modificaciones y nuevas propuestas libremente. Esta desventaja es la que tratamos de sortear proponiendo un ambiente de trabajo para desarrollar dispositivos de distintas capacidades, con una arquitectura flexible y sencilla de usar.

En este trabajo de investigación se analizan las soluciones posibles, destacando la arquitectura ARM como una de las más flexibles para hacer este tipo de desarrollos. Se analizan las herramientas de desarrollo existentes y con base en ello se propone la creación de una nueva herramienta, un simulador llamado ARMUX, pensado para dar flexibilidad y rapidez al proceso de desarrollo. Para hacer más sencillo el desarrollo se propone una capa de abstracción: el microkernel L4. Para terminar el ambiente de desarrollo completo se muestra el diseño de una tarjeta que permite ejecutar los mismos programas probados en el simulador ARMUX, y que controla también un dispositivo de video, con lo que podemos probar de una manera muy sencilla una aplicación específica creada sobre nuestro ambiente de desarrollo.

\vfill

\small

\textbf{Palabras clave:} ARM, Simulador, Sistema Integrado, Linux

\end{addmargin}

\endgroup

